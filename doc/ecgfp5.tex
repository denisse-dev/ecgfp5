\documentclass{llncs}
  
\usepackage[T1]{fontenc}
\usepackage[utf8x]{inputenc}
\usepackage[english]{babel}
\usepackage{lmodern}
\usepackage{mathtools}
%\usepackage{fullpage}
\usepackage{graphicx}
\usepackage{xspace}
\usepackage{tabularx}
\usepackage[lf]{ebgaramond}
\usepackage{biolinum}
\usepackage[cmintegrals,cmbraces]{newtxmath}
\usepackage{ebgaramond-maths}
\usepackage{multicol}
\usepackage{sectsty}
\usepackage[noend]{algpseudocode}
\usepackage{algorithm}
\usepackage{algorithmicx}
\usepackage[bottom]{footmisc}
\usepackage{caption}
%\captionsetup{font=small}
\captionsetup{labelfont={sf,bf}}
\usepackage{ragged2e}
\usepackage{xcolor}

% llncs removes generation of PDF bookmarks, we must add them back
\usepackage{etoolbox}
\makeatletter
\let\llncs@addcontentsline\addcontentsline
\patchcmd{\maketitle}{\addcontentsline}{\llncs@addcontentsline}{}{}
\patchcmd{\maketitle}{\addcontentsline}{\llncs@addcontentsline}{}{}
\patchcmd{\maketitle}{\addcontentsline}{\llncs@addcontentsline}{}{}
\setcounter{tocdepth}{3}
\makeatother
%\PassOptionsToPackage{hyphens}{url}
\usepackage{xurl}
\usepackage[pdftex,bookmarks=true]{hyperref}
\hypersetup{
    bookmarksopen=true,
    bookmarksopenlevel=3,
    bookmarksnumbered=true,
    hidelinks,
    colorlinks,
    linkcolor={red!50!black},
    citecolor={green!50!black},
    urlcolor={blue!80!black}
}
%\hypersetup{hidelinks}

\makeatletter
  % Recover some math symbols that were masked by eb-garamond, but do not
  % have replacement definitions.
  \DeclareSymbolFont{ntxletters}{OML}{ntxmi}{m}{it}
  \SetSymbolFont{ntxletters}{bold}{OML}{ntxmi}{b}{it}
  \re@DeclareMathSymbol{\leftharpoonup}{\mathrel}{ntxletters}{"28}
  \re@DeclareMathSymbol{\leftharpoondown}{\mathrel}{ntxletters}{"29}
  \re@DeclareMathSymbol{\rightharpoonup}{\mathrel}{ntxletters}{"2A}
  \re@DeclareMathSymbol{\rightharpoondown}{\mathrel}{ntxletters}{"2B}
  \re@DeclareMathSymbol{\triangleleft}{\mathbin}{ntxletters}{"2F}
  \re@DeclareMathSymbol{\triangleright}{\mathbin}{ntxletters}{"2E}
  \re@DeclareMathSymbol{\partial}{\mathord}{ntxletters}{"40}
  \re@DeclareMathSymbol{\flat}{\mathord}{ntxletters}{"5B}
  \re@DeclareMathSymbol{\natural}{\mathord}{ntxletters}{"5C}
  \re@DeclareMathSymbol{\star}{\mathbin}{ntxletters}{"3F}
  \re@DeclareMathSymbol{\smile}{\mathrel}{ntxletters}{"5E}
  \re@DeclareMathSymbol{\frown}{\mathrel}{ntxletters}{"5F}
  \re@DeclareMathSymbol{\sharp}{\mathord}{ntxletters}{"5D}
  \re@DeclareMathAccent{\vec}{\mathord}{ntxletters}{"7E}

  % Change font for algorithm label.
  \renewcommand\ALG@name{\sffamily\bfseries Algorithm}
\makeatother

% Use the sans-serif fonts (for which bold is properly defined) for
% section headings. Also ensure that subsubsections get a number and
% that the number is displayed (to distinguish them from paragraphs).
\allsectionsfont{\sffamily}
\setcounter{secnumdepth}{3}
\pagestyle{plain}

\makeatletter
\renewenvironment{abstract}{%
      \list{}{\advance\topsep by0.35cm\relax\small
      \leftmargin=1cm
      \labelwidth=\z@
      \listparindent=\z@
      \itemindent\listparindent
      \rightmargin\leftmargin}\item[\hskip\labelsep
                                    \textsf{\textbf{\abstractname}}]}
    {\endlist}
\makeatother

\spnewtheorem{mtheorem}{Theorem}{\sffamily\bfseries}{\itshape}
\spnewtheorem*{mproof}{Proof}{\sffamily\bfseries\itshape}{\rmfamily}

%\newcommand{\GF}{\mathrm{\textit{GF}}}
\newcommand{\GF}{GF}
\newcommand{\QR}{QR}
\newcommand{\bB}{\mathbb{B}}
\newcommand{\bF}{\mathbb{F}}
\newcommand{\bG}{\mathbb{G}}
\newcommand{\bN}{\mathbb{N}}
\newcommand{\bZ}{\mathbb{Z}}
\newcommand{\bR}{\mathbb{R}}
\newcommand{\neutral}{\mathbb{O}}
\newcommand{\vol}{\text{\textsf{vol}}}
\newcommand{\bitlength}{\text{\textsf{len}}}
\newcommand{\MAC}{\text{\textsf{MAC}}}
\newcommand{\MAClen}{\text{\textsf{MAClen}}}
\newcommand{\cc}{\text{\textsf{enc}}}
\newcommand{\cclen}{\text{\textsf{clen}}}
\newcommand{\Setup}{\text{\textsf{Setup}}}
\newcommand{\Eval}{\text{\textsf{Eval}}}
\newcommand{\Verify}{\text{\textsf{Verify}}}
\newcommand{\VDF}{\text{\textsf{VDF}}}
\newcommand{\VDFlen}{\text{\textsf{VDFlen}}}
\newcommand{\smod}[1]{\,\,\,(\text{mod}^{\pm} #1)}
\newcommand{\smodnospace}[1]{(\text{mod}^{\pm} #1)}

\raggedbottom

\begin{document}

\title{\textsf{EcGFp5: a Specialized Elliptic Curve}}

\author{Thomas Pornin}
\institute{NCC Group, \email{thomas.pornin@nccgroup.com}}

\maketitle
\noindent\makebox[\textwidth]{23 February, 2022}
%FIXME: adjust date when (re)publishing

\begin{abstract}
We present here the design and implementation of ecGFp5, an elliptic
curve meant for a specific compute model in which operations modulo a
given 64-bit prime are especially efficient. This model is primarily
intended for running operations in a virtual machine that produces and
verifies zero-knowledge STARK proofs. We describe here the choice of
a secure curve, amenable to safe cryptographic operations such as
digital signatures, that maps to such models, while still providing
reasonable performance on general purpose computers.
\end{abstract}

% ----------------------------------------------------------------------

\section{EcGFp5 Definition}\label{sec:intro}

Let $p = 2^{64} - 2^{32} + 1$. This is a 64-bit prime integer;
computations modulo $p$ can be relatively efficiently implemented on a
variety of platforms. It has high 2-adicity ($p-1$ is a multiple of a
large power of 2, here $2^{32}$) which makes it convenient for STARK
proofs\cite{BenBenHorRia2018}. 32-bit integer operations can also be
expressed over $\GF(p)$ since, for instance, the product of two unsigned
32-bit integers is at most $(2^{32}-1)^2$, which is lower than $p$. The
Miden VM\cite{MidenVM} is an open-source implementation of a virtual
machine whose internal opcodes work over elements of $\GF(p)$ and can
be used to generate and verify STARK proofs over arbitrary program
executions. Miden is currently funded by Polygon for blockchain-related
purposes, but it can be used in a larger spectrum of situations.
Moreover, other projects may want to use the same modulus $p$,
especially but not necessarily in conjunction with zero-knowledge
proofs. For the definition of ecGFp5, we consider the abstract and
interesting problem of defining a secure prime order group, based on an
elliptic curve, amenable to cryptographic operations such as digital
signatures, and efficiently implementable in the compute model
incarnated by the Miden VM. In that model, all elementary operations in
$\GF(p)$ have the same unit cost; this includes additions, subtractions,
multiplications, and, crucially, divisions.

The chosen curve has the following parameters. We first define the
finite field extension $\GF(p^5) = \GF(p)[z]/(z^5-3)$, i.e. the
ring of polynomials (in the symbolic variable $z$) with coefficients
in $\GF(p)$, and all operations performed modulo the polynomial $z^5-3$.
Thus, all elements can be represented as five elements of $\GF(p)$,
corresponding to the five coefficients of a polynomial of degree at
most 5. Since $z^5-3$ is irreducible over $\GF(p)$, this defines a
finite field of cardinal $p^5$.

We define an elliptic curve of equation:
\begin{equation*}
    y^2 = x(x^2 + 2x + 263z)
\end{equation*}
This is a \emph{double-odd curve}\cite{Por2020-4}: its order is $2n$
for the 319-bit prime integer:
\begin{eqnarray*}
    n &=& 106799351671714695104148491657179270274505774058 \\
      & & 1727230159139685185762082554198619328292418486241
\end{eqnarray*}
EcGFp5 is then formally defined as the group $\bG$ of points of that
curve which are \emph{not} points of $n$-torsion. The neutral element
of the group is the point $N = (0, 0)$ (the only point of order 2 on
the curve). The sum in the group of two elements $P$ and $Q$ is defined
as the curve point $P+Q+N$. As explained in \cite{Por2020-4}, this
yields a group with the proper characteristics for defining cryptographic
operations such as digital signatures or key exchange:
\begin{itemize}

    \item The group $\bG$ has prime order $n$.

    \item Elements of $\bG$ can be uniquely encoded into a field
    element; the decoding process is unambiguous and inherently
    verifies that the provided encoding was valid and canonical.

    \item Group operations can be computed with efficient complete
    formulas.

\end{itemize}

Several systems of coordinates can be used. In general, it is
recommended to use $(x, u)$ coordinates, in which $u = x/y$ for element
$(x, y)$ (for the neutral $N$, we use $u = 0$). If both $x$ and $u$ are
expressed as fractions (denoted $X/Z$ and $U/T$, respectively), then
general point addition formulas have a cost of 10 multiplications in the
field (denoted 10M), and specialized formulas for sequences of doublings
have a per-doubling cost of 2M+5S (two multiplications and five
squarings in the field). As will be detailed in the next sections,
though, within the target compute model, it is in fact more efficient to
switch to affine Weierstraß coordinates and formulas.

In the next sections, we will:
\begin{itemize}

    \item justify the choice of a degree-5 field extension;

    \item describe the implementation of field and curve operations in the
    target compute model (thereafter called ``in-VM'');

    \item formalize the choice criteria for the curve parameters;

    \item provide some details on the implementation of the curve when
    not working in the VM (i.e. the ``out-of-VM'' situation).

\end{itemize}

A copy of this paper, a test implementation in Python that emulates the
VM model to measure costs, and a reference implementation in Rust, are
provided at:
\begin{center}
    \url{https://github.com/pornin/ecgfp5/}
\end{center}

\section{Choice of Field}

Since $p$ is a 64-bit integer, we need to work in a field extension
$\GF(p^k)$ in order to have a field large enough to obtain a curve with
an adequate security. We aim at the usual ``128-bit security'' level.
For such a level, we need a field with at least a 256-bit order, hence
$k \geq 4$. Robustness of elliptic curve discrete logarithm in extension
fields has been studied in various articles. A rough summary is the
following:
\begin{itemize}

    \item If the extension degree $k$ is composite, then Weil descent
    attacks may apply, using a tower of field extensions to turn the
    problem into a discrete logarithm in an higher genus curve on a
    smaller field\cite{GauHesSma2002,AriMatNagShi2004}. To avoid such
    issues, a prime degree is highly recommended. Diem showed that if
    $k$ is prime and not lower than $11$, then such attacks cannot
    work\cite{Die2003}.

    \item A related attack using Gröbner bases was described by
    Gaudry\cite{Gau2009}; its complexity was further analyzed by Joux
    and Vitse, along with some possible variants\cite{JouVit2013}.

\end{itemize}

For performance reasons, we would like to have $k$ as small as possible.
Using $k = 4$ would allow the known attacks on quartic extension
fields\cite{AriMatNagShi2004}, with complexity about $O(p^{3/2}) \approx
2^{96}$. Though this value is quite larger than what can practically be
implemented, it still falls short of the expected ``128-bit'' level.
Thus, we need at least $k = 5$.

With $k = 5$, Gaudry's attack entails computing about $p^{2-2/5} \approx
2^{102.4}$ systems of polynomial equations, and obtaining a Gröbner
basis for each of them. Each system would contain 5 equations with 5
unknowns, and a total degree $2^{k-1} = 16$; it is expected that
obtaining the basis will require using the FGLM
algorithm\cite{FauGiaLazMor1993} with complexity $O(k D^3)$, with $k$
the number of unknowns (here, $k = 5$) and $D$ the degree of the
underlying ideal, which should be close to $2^{k(k-1)} = 2^{20}$. The
involved matrix should be mostly empty and a lower complexity might be
achieved, but even with very optimistic assumptions, it is unlikely to
go below $O(D^2) \approx 2^{40}$. This leads to a total theoretical
complexity of at least $2^{142}$, well beyond the target 128-bit level.
Joux and Vitse's variant has cost $O(C p^2)$ for some constant $C$ that
depends on $k$, again above the 128-bit level.

We can thus claim that a degree-5 extension field, $\GF(p^5)$, is
sufficient to achieve 128-bit security.

All finite fields with the same cardinal are isomorphic to each other;
we can thus choose whatever definition of that field provides the best
performance. Field extensions of degree $k$ are classically defined as
the quotient of the ring of polynomials in the base field, by a given
unitary irreducible polynomial $M$ of degree $k$. Since multiplications
in the extension field will involve reductions modulo $M$, performance
should be best if using an $M$ of minimal Hamming weight, and with
non-zero coefficients as close to 1 or -1 as possible. Among polynomials
in $\GF(p)[z]$, none of the following polynomials happens to be
irreducible: $z^5$, $z^5 \pm 1$, $z^5 \pm z^i \pm 1$ for any $i \in
[1;4]$, $z^5 \pm 2$. The next best choices are $z^5 - 3$ and $z^5 + 3$,
which are both irreducible; we thus choose $M = z^5 - 3$.

\section{In-VM Implementation}

\subsection{VM Opcodes}

We assume here that the following opcodes are offered by the target
compute model, and all have cost exactly 1 cycle:
\begin{itemize}

    \item \verb+add+, \verb+sub+, \verb+mul+ and \verb+div+ perform
    respectively addition, subtraction, multiplication and division
    in $\GF(p)$. Division fails if the divisor is zero; it is up to
    the caller to make sure that this situation does not happen.
    Negation has a specific opcode (\verb+neg+) but could also be
    implemented with a subtraction from zero.

    \item \verb+and+, \verb+or+, \verb+xor+ and \verb+not+ perform
    operations on Boolean values. A ``true'' is represented as the
    $\GF(p)$ element 1, while a ``false'' is 0. Any other value triggers
    a failure. Opcodes \verb+eq+ and \verb+neq+ compare two $\GF(p)$
    elements and return such Boolean value if the two operands, are,
    respectively, equal to each other, or different from each other.

    \item \verb+select+ applied on three values $x$, $y$ and $c$
    returns $x$ if $c = 0$, or $y$ if $c = 1$. The control value $c$
    must have a Boolean 0-or-1 value.

    \item \verb+add32+, \verb+sub32+, \verb+mul32+, \verb+div32+,
    \verb+shl32+, \verb+shr32+, and \verb+gte32+ implement operations on
    32-bit values (for addition, subtraction, multiplication, division,
    shift left, shift right, and greater-or-equal comparisons,
    respectively). Addition and subtraction are computed modulo $2^{32}$
    and have carry/borrow support for both input and output.
    Multiplication yields a 64-bit output (which always fits in a
    $\GF(p)$ element). Shifts and comparisons operate on unsigned
    values; the left shift truncates its output to 32 bits. Moreover,
    shift counts must be fixed constants. All 32-bit opcodes assume that
    the operands are in the proper range (0 to $2^{32}-1$).

\end{itemize}

The names above do not exactly match the names used by the Miden VM
assembly spe\-cification\cite{MidenAsm}; for instance, what we call here
\verb+add32+ is known as \verb+u32addc.unsafe+ in the specification
document.

The compute model should be understood as a general circuit emulation in
which only arithmetic gates have a cost, while data routing is free,
provided that it can be resolved statically. In the context of a VM
executing a program consisting of opcodes, this means that function
calls, loop control, reading from memory and writing to memory are all
free (their cost is zero); \emph{however}, this does not extend to
data-dependent conditional jumps, and array accesses at data-dependent
indexes. These operations are possible in Miden but very expensive; in
the context of this paper, we simply consider them to be forbidden. In a
sense, we use a compute model which is close to constant-time
implementations, although for different reasons.

\subsection{Field Operations}\label{sec:invm-field}

An element $x$ of $\GF(p^5)$ is represented as five coefficients $x_0$
to $x_4$, such that $x = x_0 + x_1 z + x_2 z^2 + x_3 z^3 + x_4 z^4$.
Addition and subtraction are simply done coefficient-wise; thus, an
addition in $\GF(p^5)$ boils down to five \verb+add+ opcodes, for a cost
of 5.

\paragraph{Multiplication.} Multiplication in $\GF(p^5)$ ($d\leftarrow a
+ b$) can be done in a straightforward way:
\begin{eqnarray*}
    d_0 &\leftarrow& a_0 b_0 + 3 (a_1 b_4 + a_2 b_3 + a_3 b_2 + a_4 b_1) \\
    d_1 &\leftarrow& a_0 b_1 + a_1 b_0 + 3 (a_2 b_4 + a_3 b_3 + a_4 b_2) \\
    d_2 &\leftarrow& a_0 b_2 + a_1 b_1 + a_2 b_0 + 3 (a_3 b_4 + a_4 b_3) \\
    d_3 &\leftarrow& a_0 b_3 + a_1 b_2 + a_2 b_1 + a_3 b_0 + 3 a_4 b_4 \\
    d_4 &\leftarrow& a_0 b_4 + a_1 b_3 + a_2 b_2 + a_3 b_1 + a_4 b_0
\end{eqnarray*}
The multiplications by 3 come from the reduction modulo the polynomial
$z^5 - 3$. Any constant in $\GF(p)$ other than 3 would yield the same
overall cost, except $0$, $1$ or $-1$, which would be cheaper; however,
polynomials $z^5$, $z^5+1$ and $z^5-1$ are not irreducible, and do not
yield a proper field.

Overall multiplication cost is 49. Other techniques such as Karatsuba
or Toom-Cook may reduce the number of multiplications, but at the cost
of a higher number of additions and subtractions; in our compute model,
these do not seem to provide overall cost reductions.

Squaring can be done with lower cost since, for instance, product $a_0 b_1$
and $a_1 b_0$ yield the same value when $a = b$. The cost of a squaring
operation is then 34.

\paragraph{Inversion.} Inversion in $\GF(p^5)$ can be computed very
efficiently thanks to a method first described by Itoh and
Tsujii\cite{ItoTsu1988}. Define the integer $r = 1 + p + p^2 + p^3 +
p^4$. The following holds:
\begin{equation*}
    p^5 - 1 = (p - 1) r
\end{equation*}
Therefore, for any non-zero $x$ in $\GF(p^5)$, we have:
\begin{equation*}
    x^{p^5-1} = 1 = (x^r)^{p-1}
\end{equation*}
Thus, $x^r$ is a root of the polynomial $X^{p-1}-1$. Since the roots of
that polynomial are exactly the elements of $\GF(p)$, this implies that
$x^r \in \GF(p)$ for any $x$ in $\GF(p^5)$. We can thus write the inverse
of $x$ as:
\begin{equation*}
    \frac{1}{x} = \frac{x^{r-1}}{x^r}
\end{equation*}
which can be computed as the product of $x^{r-1}$ (an element of $\GF(p^5)$)
by the inverse of $x^r$ (an element of $\GF(p)$).

Values $x^{r-1}$ and $x^r$ can furthermore be computed with appropriate
use of the Frobenius operator. Define $\phi_1(x) = x^p$ for any $x \in
\GF(p^5)$; this is a field automorphism, i.e. $\phi_1(x + y) = \phi_1(x)
+ \phi_1(y)$, and $\phi_1(xy) = \phi_1(x) \phi_1(y)$ for any $x$ and
$y$. Thus, if $x = \sum_i x_i z^i$, then:
\begin{eqnarray*}
    \phi_1(x) &=& \sum_{i=0}^4 \phi_1(x_i) z^{pi} \\
              &=& \sum_{i=0}^4 x_i (3^{i\lfloor p/5 \rfloor}) z^{i(p \bmod 5)}
\end{eqnarray*}
In our case, $p \bmod 5 = 1$, so the Frobenius operator is computed by
simply multiplying all five coefficients by precomputed constants, the
first of which being furthermore equal to 1. This is done in cost 4. We
similarly define $\phi_2(x) = \phi_1(\phi_1(x))$, which is also
computed in cost 4.

With the Frobenius operator, we compute $x^{r-1}$ as:
\begin{eqnarray*}
    x^{r-1} &=& x^{p + p^2 + p^3 + p^4} \\
            &=& \phi_1(x) \phi_1(\phi_1(x)) \phi_2(\phi_1(x) \phi_1(\phi_1(x)))
\end{eqnarray*}
which entails three Frobenius operators and two multiplications in $\GF(p^5)$.
Once $x^{r-1}$ is obtained, $x^r$ is computed by multiplying that value with
$x$; since $x^r$ is known to be in $\GF(p)$, we only need to compute the
first coefficient, with cost 10.

Inversion in $\GF(p)$ has cost 1 (this is a single \verb+div+ opcode);
however, we have to add a small corrective action to avoid a division by
zero, in case the input $x$ was equal to zero: before inverting $y = x^r$,
we compare it with zero (\verb+eq+ opcode), then add the Boolean result
(a $\GF(p)$ element with value 0 or 1) to $y$. Thus, if $x = 0$, we
end up inverting $y' = 1$ instead of $y = 0$, and the division opcode does
not fail. This corrective action has cost 2.

Final multiplication of $x^{r-1}$ by the resulting $x^{-r}$ is done in
five \verb+mul+ opcodes. The overall cost of inversion in $\GF(p^5)$ is
128 cycles (it would be 126 if the \verb+div+ opcode tolerated a divisor
equal to 0). Note that if $x = 0$, the inversion process described above
does not fail; instead, it returns 0. This is a feature; it simplifies
some operations later on.

General division is the combination of a multiplication and an inversion,
with cost 177.

\paragraph{Legendre Symbol.} We define the Legendre symbol of $x$ as
the value $x^{(p^5 - 1)/2}$; it is equal to 0 if $x = 0$, to 1 if $x$
is a non-zero quadratic residue, or $-1$ if $x$ is not a square in
$\GF(p^5)$. Again using the value $r = 1 + p + p^2 + p^3 + p^4$,
we find that:
\begin{equation*}
    x^{(p^5 - 1)/2} = (x^r)^{(p-1)/2}
\end{equation*}
Thus, the Legendre symbol of $x \in \GF(p^5)$ is equal to the Legendre
symbol of $x^r$ in $\GF(p)$. As in the case of inversion, the computation
of $x^r$ is efficient. In $\GF(p)$, we note that $(p-1)/2 = 2^{63} - 2^{31}$,
leading to the following process:
\begin{enumerate}
    \item $y \leftarrow x^r$ (result is in $\GF(p)$)
    \item $y_{31} \leftarrow y^{2^{31}}$ (with 31 \verb+mul+ opcodes)
    \item $y_{63} \leftarrow y_{31}^{2^{32}}$ (with 32 \verb+mul+ opcodes)
    \item $t \leftarrow y_{63} / y_{31}$ (one \verb+div+ opcode)
\end{enumerate}
The overall cost is 186.

\paragraph{Square Root.} We can use, again, the Frobenius operators to
speed up square roots, by noticing that, for $x\neq 0$ in $\GF(p^5)$:
\begin{eqnarray*}
    \sqrt{x} &=& \sqrt{\frac{x^r}{x^{r-1}}} \\
             &=& \frac{\sqrt{x^r}}{x^{(r-1)/2}}
\end{eqnarray*}
since $r - 1 = p + p^2 + p^3 + p^4$, which is an even integer. We can
compute $x^{(r-1)/2}$ as:
\begin{eqnarray*}
    x^{(r-1)/2} &=& x^{p(1 + p^2)(p + 1)/2} \\
                &=& \phi_1( x^{(p+1)/2} \phi_2( x^{(p+1)/2} ))
\end{eqnarray*}
We can write $x^{(p+1)/2} = x^{2^{63} + 1} / x^{2^{31}}$, allowing the
computation of that value with 63 squarings, one multiplication and one
inversion in $\GF(p^5)$. Once we obtained $x^{(p+1)/2}$, we use it to
compute $x^{(r-1)/2}$ as show above, with two Frobenius operators and
one multiplication. We can furthermore derive $x^r$ from $x^{(r-1)/2}$
with a squaring (to get $x^{r-1}$) then a multiplication by $x$; the
latter only needs to compute the lowest coefficient, since $x^r \in
\GF(p)$.

At that point, we have to compute the square root of $y = x^r$ in
$\GF(p)$. Moduli with high 2-arity are known to be inconvenient for
computing square roots. We use the following process, which really is
the Tonelli-Shanks algorithm, specialized to our compute model in which
data-dependent conditional jumps are forbidden, but multiplications are
inexpensive:
\begin{enumerate}

    \item Let $n = 32$ and $q = 2^{32} - 1$, so that $q$ is odd and
    $p = q 2^n + 1$. Let $g$ be a primitive $2^n$ root of unity in
    $\GF(p)$ (we can use $g = 7^q \bmod p$, since $7$ is a non-QR in
    $\GF(p)$). We precompute values $g_i = g^{2^i}$ for $i = 0$ to
    $n-1$.

    \item $(u, v) \leftarrow (y^{(q+1)/2}, y^q)$

    \item For $i = n-1$ down to $1$:
        \begin{enumerate}
            \item $w \leftarrow v^{2^{i-1}}$ (with $i-1$ squarings)
            \item If $w = -1 \bmod p$, then:
            $(u, v) \leftarrow (u g_{n-i-1}, v g_{n-i})$ (the new
            $(u, v)$ are always computed, but kept with \verb+select+
            opcodes only if an \verb+eq+ opcode declares that the computed
            $w$ is indeed equal to $-1$)
        \end{enumerate}

    \item If $v = 0$ or $1$, then $y$ was indeed a square, and $u$ contains
    one of its square roots. Otherwise, $y$ was not a square, and thus
    $x$ was not a square either; in that case, we arrange to set $v$ to
    zero (e.g. with an extra \verb+select+).

\end{enumerate}
The algorithm above entails $(n-1)(n-2)/2 = 450$ squarings, and some other
operations, for a total of 659 cycles for the square root in $\GF(p)$.
Combined with the computations of $x^{(r-1)/2}$ and $x^r$, and finally
combining together the value, we compute a square root in $\GF(p^5)$
in a total of 3261 cycles. The routine returns two values, the square
root itself, and a Boolean value reporting the success of the process;
if the input value was not a square, the returned ``square root'' is zero,
and the Boolean is zero.

\paragraph{Cost Summary.} We obtain the following costs for in-VM
operations in $\GF(p^5)$:
\begin{center}
    \begin{tabular}{|l|r|}
        \hline
        \textsf{\textbf{Operation}} & \textsf{\textbf{Cost (cycles)}} \\
        \hline
        addition         &    5 \\
        subtraction      &    5 \\
        multiplication   &   49 \\
        squaring         &   34 \\
        inversion        &  128 \\
        division         &  177 \\
        Legendre symbol  &  186 \\
        Square root      & 3261 \\
        \hline
    \end{tabular}
\end{center}
An important point here is that inversions in $\GF(p^5)$ are quite
inexpensive: the cost of an inversion is only about 2.57 times the
cost of a multiplication. This is not the usual situation when
dealing with elliptic curve implementations; it impacts the strategy
we will use, in particular the point addition formulas.

\subsection{Curve Formulas}\label{sec:invm-curve-formulas}

Since inversions in $\GF(p^5)$ are quite efficient (cost lower than
three times the cost of a multiplication in $\GF(p^5)$), the most
efficient formulas for computing point additions and doublings are
obtained by working with the short Weierstraß equation and affine
coordinates. Moreover, \emph{any} elliptic curve over $\GF(p^5)$ can be
expressed as a short Weierstraß curve with a suitable change of
variable; thus, no curve type will be any better or worse than any
other, efficiency-wise, for in-VM computations.

\paragraph{Change of Variable.} A double-odd curve such as ecGFp5 has
equation $y^2 = x(x^2 + ax + b)$ for two constants $a$ and $b$ (for
ecGFp5, $a = 2$ and $b = 263z$). It can be converted to the short
Weierstraß equation:
\begin{equation*}
    Y^2 = X^3 + A X + B
\end{equation*}
with constants:
\begin{eqnarray*}
    A &=& (3b - a^2)/3 \\
    B &=& a(2a^2 - 9b)/27
\end{eqnarray*}
using the following change of variable:
\begin{equation*}
    (X, Y) = (x + \frac{a}{3}, y)
\end{equation*}
This change of variable is very inexpensive: it suffices to add a single
constant to the $x$ coordinate. Moreover, that constant is in $\GF(p)$
for ecGFp5, leading to an addition in a single cycle in the VM.

EcGFp5, however, is not \emph{exactly} the elliptic curve with equation
$y^2 = x(x^2 + ax + b)$, but a subset thereof, consisting of (exactly)
the points which are not the double of any other point. In order to
perform computations on the short Weierstraß curve, we also need to map
into the subgroup of points of $n$-torsion on the short Weierstraß curve,
which entails adding the point $N$. This can be done while still on
the original equation, since $(x, y) + N = (b/x, -b y/x^2)$. Another
method, which is even simpler, is to combine the addition of $N$ with
the decoding process. Given an encoded point $w$ (nominally equal to $y/x$
for the ecGFp5 element $(x, y)$), compute the following:
\begin{enumerate}

    \item $e \leftarrow w^2 - a$

    \item $\Delta \leftarrow e^2 - 4b$

    \item $(x_1, x_2) \leftarrow ((e + \sqrt{\Delta})/2, (e - \sqrt{\Delta})/2)$
    (if $\Delta$ is not a square, then either $w = 0$, in which case the
    point is $N$ and should be decoded as the point-at-infinity in the
    short Weierstraß curve; or $w \neq 0$, and there is no solution, $w$
    is not a validly encoded point)

    \item \label{step:decode-choose-qr}If $x_1$ is a quadratic residue,
    then set $x = x_1$; otherwise, set $x = x_2$.

    \item Return $(X, Y) = (x + a/3, -wx)$

\end{enumerate}

Step~\ref{step:decode-choose-qr} is where processing diverges from
normal decoding in double-odd curves, where we would have chosen the $x$
value which is \emph{not} a quadratic residue instead. This decoding
process works because a given $w = y/x$ value is shared by two points on
the curve, $P+N$ (which is in ecGFp5) and $-P$ (which is in the subgroup
of points of $n$-torsion); here, we simply use the latter, and take the
negation into account by computing $y = -wx$ instead of $y = wx$.

It shall be noted that if $w = 0$, then the decoding above computes
$\Delta = a^2 - 4b$, which is not a quadratic residue. Moreover, in that
case, the ``point-at-infinity'' should be returned, and that point does
not have defined $(X,Y)$ coordinates. In a practical implementation, a
point in affine coordinates on the short Weierstraß curve really is a
set of \emph{three} values: the coordinates $X$ and $Y$, which are in
$\GF(p^5)$, and a Boolean flag $I$ which, when non-zero, indicates that
the point is the point-at-infinity, and the values of $X$ and $Y$ should
be ignored.

\paragraph{Point Addition.} The sum of two points $(X_1, Y_1)$
and $(X_2, Y_2)$ is the point $(X_3, Y_3)$ with:
\begin{eqnarray*}
    \lambda &=& \frac{Y_2 - Y_1}{X_2 - X_1} \\
    X_3 &=& \lambda^2 - X_1 - X_2 \\
    Y_3 &=& \lambda (X_1 - X_3) - Y_1
\end{eqnarray*}
Famously, these formulas are not complete; if $X_1 = X_2$ then either
$Y_1 = -Y_2$, in which case the sum is the point-at-infinity (sum of
a point and its opposite); or $Y_1 = Y_2$, which means that the point
is added to itself, and $\lambda$ must instead be computed as:
\begin{equation*}
    \lambda = \frac{3 X_1^2 + A}{2 Y_1}
\end{equation*}

A \emph{complete routine} that supports all cases looks like this:
\begin{enumerate}

    \item Inputs are point $(X_1, Y_1)$ (with infinity flag $I_1$) and
    $(X_2, Y_2)$ (with flag $I_2$).

    \item Compare $X_1$ with $X_2$, yielding a Boolean $s_x$ with value
    1 if they are equal, 0 otherwise. This entails five \verb+eq+ opcodes,
    and four \verb+and+ opcodes, for a cost of 9. Similarly, compare
    $Y_1$ with $Y_2$, yielding $s_y$.

    \item Set $\lambda_0$ to $Y_2 - Y_1$ if $s_x = 1$, or to $3 X_1^2 + A$
    if $s_x = 0$ (both values are computed, and \verb+select+ opcodes are
    used to keep the right one, depending on the value of $s_x$).

    \item Set $\lambda_1$ to $X_2 - X_1$ if $s_x = 1$, or to $2 Y_1$ if
    $s_x = 0$.

    \item $\lambda \leftarrow \lambda_0 / \lambda_1$

    \item $X_3 \leftarrow \lambda^2 - X_1 - X_2$

    \item $Y_3 \leftarrow \lambda (X_1 - X_3) - Y_1$

    \item $I_3 \leftarrow s_x \text{\textsc{\ and\ }} s_y$

    \item If $I_1 \neq 0$, then replace $(X_3, Y_3, I_3)$ with $(X_2, Y_2, I_2)$

    \item If $I_2 \neq 0$, then replace $(X_3, Y_3, I_3)$ with $(X_1, Y_1, I_1)$

\end{enumerate}

This routine handles all edge cases (point doublings, point-at-infinity
as input or output operand) with a fixed cost of 387 cycles in the VM.
This cost is about 7.9 times the cost of a single multiplication: this
is faster than the best know curve point adding formulas that do not
involve any inversion.

Optimizations are possible in some cases:
\begin{itemize}

    \item When it is \emph{a priori} known that the addition is not an
    edge case, then we can avoid in particular the squaring involved in
    computing $\lambda_0$ for a point doubling, and all the \verb+select+
    opcodes, leading to a routine with cost 290.

    \item For explicit point doublings, the \verb+select+ opcodes can also
    be skipped, since the double of a non-infinity $n$-torsion point cannot
    be the point-at-infinity; we can simply keep the infinity flag
    unchanged, and apply the doubling formulas on the coordinates. This
    leads a point doubling in 326 cycles.

\end{itemize}

\paragraph{Point Multiplication.} In the specific case of the
multiplication of a point $P$ by a scalar $v$, the following process can
be used:
\begin{enumerate}

    \item Reduce the scalar $v$ modulo $n$, then add $n$ to get a value
    $k$ in the $n$ to $2n-1$ range.

    \item Split the scalar into chunks $\lceil 321/w \rceil$ of $w$ bits,
    for a given window width $w$ (experimentally, $w = 4$ seems to be
    the best choice here). Each chunk $c_i$ yields a signed digit $d_i$
    with the following:
    \begin{enumerate}

        \item Initialize a carry $m$ to 0.

        \item For each chunk $c_i$ (in least-to-most significant order),
        add $m$ to the value of $c_i$. If $m + c_i \leq 2^{w-1}$, then
        set $d_i$ to that value, and set $m$ to zero; otherwise, set
        $d_i$ to $m + c_i - 2^w$, and set $m$ to 1.

    \end{enumerate}
    All digits are between $-2^{w-1} + 1$ and $+2^{w - 1}$; moreover,
    with the chosen parameters, it can be shown that the top digit is
    necessarily greater than or equal to 1. Since the VM works with
    \emph{unsigned} 32-bit integers (mapped to $\GF(p)$ elements), the
    sign and absolute value of each digit may be returned separately.

    \item Fill an array $W$ with points $i P$ for $i = 1$ to $2^{w-1}$.
    This is called the ``window''. For an index value $d$ between $-2^{w-1}$
    and $+2^{w-1}$, the point $d P$ can be recovered with a lookup
    process (detailed below).

    \item For all digits $d_i$, starting with the second-to-top digit
    down to $d_0$:
    \begin{enumerate}

        \item Multiply $Q$ by $2^w$ with $w$ successive doublings.

        \item Lookup point $d_i P$ from the window, and add it to $Q$.

    \end{enumerate}

\end{enumerate}

It can be shown that in this process, if input $P$ was not the
point-at-infinity, then for all digits except the last two ($d_1$ and
$d_0$), the addition of the looked up point $d_i P$ to the current $Q$
cannot be an edge case of the point addition formulas; thus, each of
these additions can be the specialized routine with cost 290; only the
last two iterations need to use the generic routine with cost 387.

All these operations, including the building of the window, can be
performed assuming that the input $P$ is not the point-at-infinity; it
suffices to combine (with a Boolean \textsc{or}) the initial flag $I_P$
with the current flag $I_Q$ to obtain a proper result even in case the
input $P$ is the point-at-infinity. Note that window building can then
use the specialized point addition, and the infinity flag of each window
element needs not be stored.

The window lookup for digit value $d$ is done as follows:
\begin{enumerate}

    \item Initialize variables $X$ and $Y$ to copies of $X_P$ and $Y_P$.

    \item For indices $i = 2$ to $2^{w-1}$, replace $X$ and $Y$ with
    the coordinates of $iP$ (from the window) if and only if $|d| = i$.

    \item If $d < 0$, then replace $Y$ with $-Y$.

    \item Set the flag $I$ to 1 if $d = 0$, or to 0 otherwise.

\end{enumerate}
Lookup cost increases with window size, with an overhead of 11 cycles
per extra point.

Overall cost of the point multiplication function, including the
addition of $n$ to $v$ and the split into digits, and the building
of the window, was measured to be 138482 cycles.

\paragraph{Key Pair and Signature Generation.} A special case of point
multiplication is when the point to multiply is the conventional
generator point $G$. This is the main curve operation in public/private
key pair generation, as well as signature generation. In that case:
\begin{itemize}

    \item The window can be precomputed.

    \item Since there is no cost for window building, a larger window
    may offer better performance, although lookup costs will dominate
    with large windows.

    \item Several windows for precomputed multiples of $G$ may be used
    conjointly, in order to reduce the number of loop iterations and
    doubles. For instance, with 8 windows for all $2^{40i} G$ (with
    $i = 0$ to 7), 7/8th of the point doublings can be avoided,
    leading to a considerable speed-up.

\end{itemize}

There are many possible trade-offs between window size, number of
windows, and code size. When generating or verifying STARK proofs, the
whole implementation can be conceptually unrolled, and large tables of
constants used; other situations might call for more compact
implementations.

\paragraph{Signature Verification.} Verification of a Schnorr signature
entails checking that $d G + e Q = R$ for some scalars $d$ and $e$
(obtained from the signature itself, and some hashing), conventional
generator $G$, public key $Q$, and the point $R$ whose encoding is the
first half of the signature. In general, there are many optimizations
that can be applied on signature verification:
\begin{itemize}

    \item The size of the two scalars can be about halved by using the
    Antipa \emph{et al} method\cite{AntBroGalLamStrVan2005}, usually
    with Lagrange's algorithm for lattice reduction\cite{Por2020-2}.

    \item Scalar representation as signed digits can use the w-NAF
    representation, in which most digits are zero and all the non-zero
    digits are odd, leading to fewer and faster window lookups.

    \item Several verifications can be performed simultaneously with
    a randomized batch verification process, allowing important cost
    sharings.

\end{itemize}

Unfortunately, most of these optimizations require conditional execution
of some kind, depending on the involved data. This is usually not a
problem (signature verification nominally happens only over public
data), but in the VM compute model, conditional execution is quite
inconvenient.

Computation of $d P + e Q$ for two points $P$ and $Q$, and two scalars
$d$ and $e$, can still use Straus's algorithm\cite{Str1964} (often known
in cryptography as ``Shamir's trick'') to mutualize point doublings,
i.e. perform about 320 doublings in total, instead of 640 as would be
obtained with two seperate point multiplication operations.

\section{Curve Parameters Selection}

As we saw in section~\ref{sec:invm-curve-formulas}, in-VM curve
operations will preferably use a short Weierstraß equation and affine
formulas. \emph{Any} elliptic curve is amenable to such a
representation; moreover, the specific values of the Weierstraß
constants $A$ and $B$ have little to no incidence on in-VM performance:
$A$ is only used in an addition over $\GF(p^5)$ as part of the doubling
formulas, and $B$ is not used at all in the formulas. Thus, the in-VM
compute model does not imply any constraint on the curve equation type
we will use. We are free to select a curve type that favours out-of-VM
performance, as long as it provides the required security
characteristics.

For proper security in arbitrary protocols, we need a prime-order group
with a unique, canonical and verifiable encoding\cite{CreJac2019}. In
practice, this restricts our choice to the following:
\begin{itemize}

    \item A curve with a prime order. This requires using the generic
    short Weierstraß equation. Encoding output consists of the $x$
    coordinate, along with a single ``sign'' bit for $y$ to designate
    which square root of $y^2$ is intended.

    \item A double-odd curve\cite{Por2020-4}, with order $2n$ for a
    prime $n$. An element of the prime order group is encoded as
    a single field element.

    \item A Montgomery or twisted Edwards curve, of order $4n$ or $8n$
    for a prime $n$, along with the Decaf/Ristretto encoding
    process\cite{Ham2015,RistrettoWeb}.

\end{itemize}

All three kinds offer division-less complete formulas for safe out-of-VM
processing. The formulas for prime-order short Weierstraß
curves\cite{RenCosBat2015} are somewhat slower than for the other two
(12M for general addition, 8M+3S for doubling). Twisted Edwards curves
have the fastest general addition formulas (8M), and point doubling with
cost 4M+4S (an alternate choice of representation called ``inverted
coordinates'' offers addition in 9M+1S and doubling in 3M+4S). However,
double-odd curves have better doubling formulas (2M+5S per-doubling
overhead); the encoding/decoding process of double-odd curves is also
somewhat simpler than that of Decaf/Ristretto, and allows for efficient
validation that a point is decodable and canonical (with only a Legendre
symbol). We will thus choose a double-odd curve.

There are several representations for double-odd curves; in general, it
is recommended to use fractional $(x, u)$ coordinates, for which
complete formulas are known. A point is represented as a quadruplet
$(X{:}Z{:}U{:}T)$, which is such that $x = X/Z$ and $u = x/y = U/T$.
Point addition formulas on that representation entail some
multiplications by the equation constants $a$ and $b$, and also the
constants $\alpha = (4b - a^2)/(2b - a)$ and $\beta = (a - 2)/(2b - a)$.
Choosing $a = 2$ implies that $\alpha = 2$ and $\beta = 0$, which is
convenient.

If $a = 2$, which is an element of $\GF(p)$ (the base field), we must
choose $b$ outside of $\GF(p)$; otherwise, the curve over the base
field would be a subgroup of the curve over $\GF(p^5)$, and it would
not be possible to obtain total curve order $2n$ for a prime $n$. To
speed up multiplications by $b$, we will still want to use a constant
$b$ equal to $b_i z^i$ for some $i \in [1;4]$, and $b_i$ as small as
possible as an integer (in absolute value). We thus use the following
search process:
\begin{enumerate}

    \item $c \leftarrow 1$

    \item \label{step:find-curve-loop}For $i = 1$ to $4$:
    \begin{enumerate}

        \item If curve $y^2 = x(x^2 + 2x + c z^i)$ has order $2n$ with
        $n$ prime, then return $b = c z^i$.

        \item If curve $y^2 = x(x^2 + 2x - c z^i)$ has order $2n$ with
        $n$ prime, then return $b = -c z^i$.

    \end{enumerate}

    \item $c \leftarrow c + 1$

    \item Loop to step~\ref{step:find-curve-loop}.

\end{enumerate}

As explained in~\cite{Por2020-4}, a curve of equation
$y^2 = x(x^2 + ax + b)$ may be double-odd (i.e. order $2n$ for an odd
integer $n$) only if neither $b$ nor $a^2 - 4b$ is a quadratic residue;
this gives a fast test that allows skipping most of the expensive
point counting operations in the process described above.

Using the process above, the first usable curve is obtained for
$b = 263z$. This yields the curve whose parameters were given
in section~\ref{sec:intro}.

While the curve selection process is not known to induce any bias or
select a curve with uncommon properties, we still checked that its
embedding degree is large. For a curve defined over a finite field
$\GF(q)$ and with a subgroup of prime order $n$ (that does not divide
$q$), the embedding degree is the smallest integer $e > 0$ such that $n$
divides $q^e - 1$ (or, said otherwise, $e$ is the multiplicative order
of $q$ modulo $n$). If $e$ is very small, then the Weil, Tate and
similar pairings can be computed, reducing the discrete logarithm in the
curve into the discrete logarithm over the multiplicative group of
invertible elements in $\GF(q^e)$. A randomly selected curve should have
a very large embedding degree $e$ (about the same size as $n$); a low
embedding degree does not necessarily imply a weakness (except if $e$ is
so small that the discrete logarithm in $\GF(q^e)$ can be computed more
efficiently than in the curve itself), but it would hint at some
unexpected internal structure. In the case of ecGFp5, we checked that
$e = (n - 1)/5$, i.e. a 317-bit integer, close to the size of $n$ itself,
as is expected of a randomly selected curve. To perform this check, notice
that $e$ is necessarily a divisor or $n - 1$; thus, we can factor $n - 1$
to try all values $(n-1)/r$ for any prime $r$ that divides $n - 1$. The
factorization of $n - 1$ is:
\begin{eqnarray*}
    n - 1 &=& 2^5 \cdot 5 \cdot 163 \cdot 769 \cdot 1059871 \\
          & & \cdot\ 253243826720162431254857814100127 \\
          & & \cdot\ 198400523053184002814403536918162724916343842520561
\end{eqnarray*}

\section{Out-of-VM Implementation}

EcGFp5 was designed to match the abilities of the target VM compute
model, not the abilities of any specific concrete CPU such as modern x86
or ARM. Out-of-VM performance is thus expected to be lower than what
is usually expected from fast elliptic curves with 128-bit security,
for two reasons:
\begin{itemize}

    \item Operations in $\GF(p)$ have some overhead. The specific value
    of $p = 2^{64} - 2^{32} + 1$ allows for some fast reduction
    techniques, but fast reduction is still more expensive than no
    reduction at all.

    \item $\GF(p^5)$ is a 320-bit field, $1.25$ times larger than a
    256-bit field. Point multiplication has a cost cubic in the size
    of the field (with commonly used field sizes); thus, as a very
    rough approximation, we should expect a cost factor
    $1.25^3 \approx 1.95$ compared to a usual curve with 128-bit
    security.

\end{itemize}

We implemented ecGFp5 in the Rust programming language. The
implementation is constant-time and efficient; it is nonetheless fully
portable (it uses only the \verb+core+ library; it includes no inline
assembly, no architecture-specific intrinsics, and no \verb+unsafe+
code). We still achieve the following performance on an Intel i5-8259U
``Coffee Lake'' CPU (using Rust compiler version 1.57.0, with
extra flags ``-C target-cpu=native'' to allow the compiler to use
opcodes available on that CPU, in particular \verb+mulx+):
\begin{center}
    \begin{tabular}{|l|r|}
        \hline
        \textsf{\textbf{Operation}} & \textsf{\textbf{Cost (cycles)}} \\
        \hline
        $\GF(p)$ addition                &      4.02 \\
        $\GF(p)$ subtraction             &      3.02 \\
        $\GF(p)$ multiplication          &     10.18 \\
        $\GF(p)$ inversion               &    737.39 \\
        $\GF(p)$ Legendre symbol         &    714.20 \\
        $\GF(p)$ Square root             &   5430.19 \\
        \hline
        $\GF(p^5)$ addition              &      8.80 \\
        $\GF(p^5)$ subtraction           &      5.78 \\
        $\GF(p^5)$ multiplication        &     94.03 \\
        $\GF(p^5)$ squaring              &     68.63 \\
        $\GF(p^5)$ inversion             &   1069.75 \\
        $\GF(p^5)$ Legendre symbol       &   1042.20 \\
        $\GF(p^5)$ Square root           &  12410.38 \\
        \hline
        ecGFp5 point addition            &   1328.50 \\
        ecGFp5 point doubling            &    985.37 \\
        ecGFp5 point doubling $\times 5$ &   3971.39 \\
        ecGFp5 point multiplication      & 363168.03 \\
        ecGFp5 generator multiplication  & 109516.20 \\
        ecGFp5 mul+add verification      & 336952.88 \\
        \hline
    \end{tabular}
\end{center}

In this table, an average per-operation time is reported for a long
sequence of dependent operations: the output of each operation is used
as part of the input to the next one. When some operations do not depend
on each other, they may be run concurrently by the CPU to some extent;
this is how, for instance, a subtraction in $\GF(p)$ costs 3 cycles, but
a subtraction in $\GF(p^5)$ can be done in less than 6 cycles instead of
15.

``Point doubling $\times 5$'' means five successive point doublings. The
fractional $(x, u)$ formulas on double-odd curves include optimizations
for long sequences of successive doublings, for a cost of
2M+1S+$j$(2M+5S) for $j$ doublings; this is why that sequence cost
is only about four times the cost of a single doubling, instead of five
times.

\subsection{Field Operations}

There are several possible representations of integers modulo $p$. In
our implementation, we chose to use \emph{Montgomery representation}: an
element $x \in \GF(p)$ is represented as $2^{64}x \bmod p$, normalized
as an integer in the $[0;p-1]$ range. This representation is coupled
with \emph{Montgomery multiplication}, which, given $x$ and $y$ in
$\GF(p)$, computes $xy/2^{64} \bmod p$. Hence, the Montgomery
multiplication of $2^{64}x$ and $2^{64}y$ yields $2^{64}xy$, which is
the Montgomery representation of the product $xy$. The cornerstone of
this support is the reduction function, which, given a 128-bit input $x$
(lower than $2^{64}p = 2^{128} - 2^{96} + 2^{64}$) returns $x/2^{64}
\bmod p$. This reduction is implemented in Rust as follows:
\begin{verbatim}
    const fn montyred(x: u128) -> u64 {
        let xl = x as u64;
        let xh = (x >> 64) as u64;
        let (a, e) = xl.overflowing_add(xl << 32);
        let b = a.wrapping_sub(a >> 32).wrapping_sub(e as u64);
        let (r, c) = xh.overflowing_sub(b);
        r.wrapping_sub(0u32.wrapping_sub(c as u32) as u64)
    }
\end{verbatim}
This reduction works as follows:
\begin{itemize}

    \item The input $x$ is split into two 32-bit words and one 64-bit
    word: $x = x_0 + 2^{32} x_1 + 2^{64} x_2$ with $x_0$ and $x_1$
    being unsigned 32-bit integers, and $x_2$ being 64-bit. Note that
    the assumption on the range of the function input implies that
    $x_2 < p$.

    \item The third function line adds $2^{32} x_0$ to
    $2^{32} x_1 + x_0$, with a carry in $e$. We thus obtain:
    \begin{equation*}
        a = -2^{64}e + 2^{32} (x_0 + x_1) + x_0
    \end{equation*}

    \item On the fourth line, we compute:
    \begin{eqnarray*}
        b &=& -2^{64}e + 2^{32} (x_0 + x_1) + x_0 + 2^{32}e - (x_0 + x_1) - e \\
          &=& 2^{32} (x_0 + x_1) - x_1 - ep \\
          &=& 2^{32} x_0 + (2^{32} - 1) x_1 - ep
    \end{eqnarray*}
    If $x_0 + x_1 < 2^{32}$, then $e = 0$ and this value is bounded as:
    \begin{equation*}
        0 \leq b = 2^{32} x_0 + (2^{32} - 1) x_1 = x_0 + (2^{32} - 1)(x_0 + x_1)
        \leq 2^{32} - 1 + (2^{32} - 1)^2 < p
    \end{equation*}
    Otherwise, if $x_0 + x_1 \geq 2^{32}$, then $e = 1$, and:
    \begin{equation*}
        0 = 2^{64} - (2^{32} - 1) - p \leq 2^{32}(x_0 + x_1) - x_1 - p
        \leq 2^{32}(2^{33} - 2) - 1 - p < p
    \end{equation*}
    In both cases, the computed value indeed fits in the variable \verb+b+
    without truncation, and we know that $b < p$.

    \item Since $1 = 2^{32} - 2^{64} \bmod p$, we know that:
    \begin{eqnarray*}
        x_0 + 2^{32} x_1 &=& 2^{32} x_0 - 2^{64} x_0
                             + 2^{64} x_1 - 2^{96} x_1 \mod p \\
                         &=& 2^{64} x_0 - 2^{96} x_0 - 2^{64} x_0
                             + 2^{64} x_1 - 2^{96} x_1 \mod p \\
                         &=& -2^{64} (2^{32} (x_0 + x_1) - x_1) \mod p \\
                         &=& -2^{64} b \mod p
    \end{eqnarray*}
    Therefore, $x/2^{64} = x_2 - b \bmod p$. At that point, we have both
    $x_2$ (in \verb+xh+) and $b$ (in \verb+b+), and both are in the
    $[0;p-1]$ range; thus, we only have to do a single subtraction, and
    adding back $p$ in case that would yield a negative result. This is
    what is done in the last two lines of the function. Specifically,
    a subtraction is done with result in \verb+r+ and borrow flag in
    \verb+c+. If the borrow is 1, then we subtract it from zero with
    a 32-bit wrapping operation, which then yields
    $2^{32}-1 = -p \bmod 2^{64}$. We then subtract that value from
    \verb+r+, which is equivalent to adding $p$ (modulo $2^{64}$). It
    is easily seen that if there were no borrow (\verb+c+ is zero),
    then the last line does not change the value of \verb+r+. In both
    cases, the correctly reduced output value is computed.

\end{itemize}

Using Montgomery representation has the benefit of producing stricly
normalized values in $[0;p-1]$ range, which allows fast subtractions and
additions. Subtraction modulo $p$ is implemented just like the last two
lines of the reduction function, described above; it has an expected
latency of 3 cycles, which is corroborated by the benchmarks.

Montgomery multiplication is a $64\times 64\rightarrow 128$
multiplication followed by Montgomery reduction. On recent Intel x86
CPUs, this multiplication uses the \verb+mulx+ opcode, which yields the
low half of the result in 3 cycles (4 cycles for the upper half).
Following operations involved in the Rust code above, we may expect that
an optimal implementation will yield the result in a total of 10 cycles
(assuming that all non-compute data movements are ``free'', as well as
the zero-extension of a 32-bit value to 64 bits); again, this is what we
achieve in the benchmarks.

For multiplications and squarings in $\GF(p^5)$, it is beneficial to
mutualize Montgomery reductions. For instance, the computation of the
low coefficient of a product is performed as follows:
\begin{verbatim}
    fn mul_to_k0(&self, rhs: &Self) -> GFp {
        let pp0 = (self.0[0].0 as u128) * (rhs.0[0].0 as u128);
        let pp1 = (self.0[1].0 as u128) * (rhs.0[4].0 as u128);
        let pp2 = (self.0[2].0 as u128) * (rhs.0[3].0 as u128);
        let pp3 = (self.0[3].0 as u128) * (rhs.0[2].0 as u128);
        let pp4 = (self.0[4].0 as u128) * (rhs.0[1].0 as u128);
        let zhi = (pp0 >> 64) + 3 * ((pp1 >> 64)
            + (pp2 >> 64) + (pp3 >> 64) + (pp4 >> 64));
        let zlo = ((pp0 as u64) as u128) + 3 * (
            ((pp1 as u64) as u128)
            + ((pp2 as u64) as u128)
            + ((pp3 as u64) as u128)
            + ((pp4 as u64) as u128));
        GFp(GFp::montyred(zlo + (zhi << 32) - zhi))
    }
\end{verbatim}

We recognize the five products (for $a_0 b_0$, $a_1 b_4$,...), each
yielding a 128-bit output. The full linear combination could be up to
132 bits in length, which exceeds what can be stored in a \verb+u128+
variable. To keep to sizes for which Rust has portable types, and to
avoid some contention on carry flags, we do the linear combination
twice, on the low and high halves separately; the final expression which
assembles \verb+zlo+ and \verb+zhi+ is a partial reduction (using the
fact that $2^{64} = 2^{32} - 1 \bmod p$) which outputs a value that fits
on 100 bits, well in range of the Montgomery reduction function.

For inversions, Legendre symbols and square roots, methods described
in section~\ref{sec:invm-field} still apply, but since divisions in
$\GF(p)$ are much more expensive than multiplications (in out-of-VM
architectures), inversions in $\GF(p^5)$ are not as fast as inside the
VM. We obtain an inversion in about $11.4$ times the cost of a
multiplication in $\GF(p^5)$, which is very fast compared with the
situation in prime fields, but still not fast enough to contemplate
use of affine coordinates on the short Weierstraß curve.

\subsection{Curve Operations}

Using fractional $(x, u)$ coordinates, we obtain generic point
addition in 10M, and generic point doubling in 4M+5S; however, some
optimizations can be applied to sequences of doublings, making it
worthwhile to organize operations such that doublings happen in such
long sequences.

We use window optimizations, similar to in-VM operations (see
section~\ref{sec:invm-curve-formulas}). Inversions in $\GF(p^5)$ are
fast enough to allow normalization of window points to affine
coordinates; mixed addition (addition between a point in fractional $(x,
u)$ and a point in affine $(x, u)$ coordinates) has cost 8M instead of
10M. Conversion of $t$ points from fractional to affine coordinates
involves inverting $2t$ field elements, which can be done in a single
inversion and $3(2t-1)$ multiplications in $\GF(p^5)$ (using
Montgomery's trick of computing $1/u$ and $1/v$ as $(1/uv)v$ and
$(1/uv)u$, respectively, and applying it recursively). With a 5-bit
window, 64 point additions are needed, and using affine points saves 128
multiplications in $\GF(p^5)$, while the normalization involves one
inversion and 93 extra multiplications, making the operation worthwhile.
Moreover, using affine coordinates for window points makes these points
smaller in RAM, which speeds up constant-time window lookups. It is
expected that the optimal window size will be 4 or 5 bits, depending
on the target architecture; on the test x86 system, 5-bit windows seem
slightly better.

For the special case of multiplying the conventional generator, multiple
windows can be used (in our implementation, eight windows are used, for
$G$, $2^{40} G$, $2^{80} G$,...) and they are precomputed, thereby
avoiding the cost of conversion to affine. This operation is used when
generating a new key pair, and when producing a Schnorr signature.

For signature verification, we apply the Antipa et al
optimization\cite{AntBroGalLamStrVan2005} with Lagrange's lattice
reduction algorithm\cite{Por2020-2}; the latter algorithm is implemented
in about 20400 cycles on average. Verification is nominally on public
values, and thus needs not be constant-time; we use direct array
accesses for lookups. However, we do \emph{not} use w-NAF representation
of scalars, because a regular addition schedule favours long sequences
of sequential doublings, for which performance is better than isolated
doublings.

\section{Conclusion}

We presented an elliptic curve designed for a specific compute model.
Although we use the Miden VM as a representative of that model, we
expect this curve to be generally useful for other projects related to
zero-knowledge proofs; curve design and implementation is also an
interesting problem in its own right. As a fully general-purpose curve,
ecGFp5 performance is not on par with the fastest standard curves (e.g.
Curve25519), but is still decent enough: a single core on a laptop
computer or a smartphone can generate or verify thousands of signatures
per second.

\section*{Acknowledgements}

We thank Bobbin Threadbare and Hamish Ivey-Law for providing the target
context and useful discussions on optimized implementations in
$\GF(p^5)$, and Pierrick Gaudry for pointers and explanations on the
fine details of curve attacks in extension fields.

%\newpage
\begin{thebibliography}{20}

% Ideally we would do that only for URLs, but I don't know how to do
% that in a non-painful way.
%\RaggedRight

\bibitem{AntBroGalLamStrVan2005}
A.~Antipa, D.~Brown, R.~Gallant, R.~Lambert, R.~Struik and S.~Vanstone,
\emph{Accelerated Verification of ECDSA signatures},
Selected Areas in Cryptography - SAC 2005, Lecture Notes in Computer
Science, vol~3897, pp.~307-318, 2005.

\bibitem{RistrettoWeb}
T.~Arcieri, I.~Lovecruft and H.~de Valence,
\emph{The Ristretto Group},\\
\url{https://ristretto.group/}

\bibitem{AriMatNagShi2004}
S.~Arita, K.~Matsuo, K.~Nagao and M.~Shimura,
\emph{A Weil Descent Attack against Elliptic Curve Cryptosystems over
Quartic Extension Fields},
IEICE Transactions on Fundamentals of Electronics, Communications and
Computer Sciences, vol.~E89-A, issue.~5, 2006.

\bibitem{BenBenHorRia2018}
E.~Ben-Sasson, I.~Bentov, Y.~Horesh and M.~Riabzev,
\emph{Scalable, transparent, and post-quantum secure computational integrity},\\
\url{https://eprint.iacr.org/2018/046}

\bibitem{CreJac2019}
C.~Cremers and D.~Jackson,
\emph{Prime, Order Please! Revisiting Small Subgroup and Invalid Curve
Attacks on Protocols using Diffie-Hellman},
IEEE 32nd Computer Security Foundations Symposium (CSF), 2019.

\bibitem{Die2003}
C.~Diem,
\emph{The GHS attack in odd characteristic},
Journal of the Ramanujan Mathematical Society, vol.~18, issue~1,
pp.~1-32, 2003.

\bibitem{FauGiaLazMor1993}
J.-C.~Faugère, P.~Gianni, D.~Lazard and T.~Mora,
\emph{Efficient Computation of Zero-dimensional Gröbner Bases by Change
of Ordering},
Journal of Symbolic Computation, vol.~16, issue~4, pp.~329-344, 1993.

\bibitem{Gau2009}
P.~Gaudry,
\emph{Index calculus for abelian varieties of small dimension and the
elliptic curve discrete logarithm problem},
Journal of Symbolic Computation, vol.~44, issue~12, pp.~1690-1702, 2009.

\bibitem{GauHesSma2002}
P.~Gaudry, F.~Hess and N.~Smart,
\emph{Constructive and destructive facets of Weil descent on elliptic curves},
Journal of Cryptology, vol.~15, issue~1, pp.~19-46, 2002.

\bibitem{Ham2015}
M.~Hamburg,
\emph{Decaf: Eliminating cofactors through point compression},
Advances in Cryptology - CRYPTO 2015, Lecture Notes in Computer Science,
vol.~9215, pp.~705-723, 2015.

\bibitem{ItoTsu1988}
T.~Itoh and S.~Tsujii,
\emph{A Fast Algorithm for Computing Multiplicative Inverses in
$\GF(2^m)$ Using Normal Bases},
Information and Computation, vol.~78, pp.~171-177, 1988.

\bibitem{JouVit2013}
A.~Joux and V.~Vitse,
\emph{Elliptic curve discrete logarithm problem over small degree
extension fields. Application to the static Diffie-Hellman problem on
$\mathbb{F}_q^5$},
Journal of Cryptology, vol.~26, issue~1, pp.~119-143, 2013.

\bibitem{MidenVM}
\emph{Polygon Miden},\\
\url{https://github.com/maticnetwork/miden}

\bibitem{MidenAsm}
\emph{Miden Assembly},\\
version 0.2, accessed on 2022-02-21,\\
\url{https://hackmd.io/YDbjUVHTRn64F4LPelC-NA}

\bibitem{Por2020-2}
T.~Pornin,
\emph{Optimized Lattice Basis Reduction In Dimension 2, and Fast Schnorr
and EdDSA Signature Verification},\\
\url{https://eprint.iacr.org/2020/454}

\bibitem{Por2020-4}
T.~Pornin,
\emph{Double-Odd Elliptic Curves},\\
\url{https://eprint.iacr.org/2020/1558}

\bibitem{RenCosBat2015}
J.~Renes, C.~Costello and L.~Batina,
\emph{Complete addition formulas for prime order elliptic curves},
Advances in Cryptology – Eurocrypt 2016, Lecture Notes in Computer Science,
vol.~9665, pp.~403-428, 2016.

\bibitem{Str1964}
E.~Straus,
\emph{Addition chains of vectors (problem 5125)},
American Mathematical Monthly, vol.~70, pp.~806-808, 1964.

\end{thebibliography}

%\appendix

\end{document}
